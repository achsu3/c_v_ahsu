%%%%%%%%%%%%%%%%%%%%%%%%%%%%%%%%%%%%%%%%%
% Important note:
% This template requires the resume.cls file to be in the same directory as the
% .tex file. The resume.cls file provides the resume style used for structuring the
% document.
%
%%%%%%%%%%%%%%%%%%%%%%%%%%%%%%%%%%%%%%%%%

%----------------------------------------------------------------------------------------
%	PACKAGES AND OTHER DOCUMENT CONFIGURATIONS
%----------------------------------------------------------------------------------------

\documentclass{resume} % Use the custom resume.cls style

\usepackage{textcomp}

\usepackage[left=0.75in,top=0.6in,right=0.75in,bottom=0.6in]{geometry} % Document margins
\newcommand{\tab}[1]{\hspace{.2667\textwidth}\rlap{#1}}
\newcommand{\itab}[1]{\hspace{0em}\rlap{#1}}
\name{Amanda C. Hsu} % Your name
 \address{achsu3.github.io} % Your address
%\address{123 Pleasant Lane \\ are City, State 12345} % Your secondary addess (optional)
%\address{847-530-7515  \\ ahsu67@gatech.edu} % Your phone number and email

\begin{document}
	
	%----------------------------------------------------------------------------------------
	%	EDUCATION SECTION
	%----------------------------------------------------------------------------------------
	
	\begin{rSection}{Education}
		%--copy and paste this region  if you need more--
		{\bf Georgia Institute of Technology} \hfill {\em Expected Graduation: 2026} 
		\\ Ph.D. in Computer Science\hfill \\
		Advisors: Professor Paul Pearce, Professor Frank Li\\
		\\
		{\bf University of Illinois at Urbana-Champaign} \hfill {\em May 2021} 
		\\ B.S. Computer Engineering, with Honors\hfill %{\em GPA: 3.52} 
		\\ Senior Thesis: \textit{Exploring Boundaries Between Organizations via IPv4 Scan Data}
	\end{rSection} 
	
	\begin{rSection}{Publications}
		%--copy and paste this region  if you need more--
		\begin{enumerate}
			\item  \textbf{Amanda Hsu}, Paul Pearce, Frank Li. Characterizing Address Structure and Behavior in Active IPv6 Networks. \textit{Currently under review}

			\item  \textbf{Amanda Hsu}, Frank Li, Paul Pearce, Oliver Gasser. A First Look At NAT64 Deployment In-The-Wild. Passive and Active Measurement Conference (PAM) 2024 \textit{to appear}
			
			\item  \textbf{Amanda Hsu}, Frank Li, Paul Pearce. Fiat Lux: Illuminating IPv6 Apportionment with Different Datasets. 2023 ACM SIGMETRICS.
			
			\item  Mohammad A. Noureddine, Ahmed M. Fawaz, \textbf{Amanda Hsu}, Cody Guldner, Sameer Vijay, Tamer Başar, William H. Sanders. Revisiting Client Puzzles for State Exhaustion Attacks Resilience. 2019 49th Annual IEEE/IFIP International Conference on Dependable Systems and Networks (DSN).
			
			\item Mohammad A. Noureddine, \textbf{Amanda Hsu}, Matthew Caesar, Fadi A. Zaraket, William H. Sanders, P4 AIG: Circuit-Level Verification of P4 Programs. 2019 49th Annual IEEE/IFIP International Conference on Dependable Systems and Networks (DSN).
			
		\end{enumerate}
		
		%--copy and paste this region  if you need more--
	\end{rSection}
	%----------------------------------------------------------------------------------------
	%	EXPERIENCE SECTION
	%----------------------------------------------------------------------------------------
	\begin{rSection}{Research Experience}
		
		{\bf Measuring Semantic Embeddings IPv6 Networks} \hfill{\em October 2022 - Present}\\
		{\em Georgia Institute of Technology}\\
		{Advisors: Professor Paul Pearce, Professor Frank Li}
		\begin{itemize}
			\item Characterized IPv6 networks according to addressing patterns with respect to responsiveness to different ports, protocols, and application-level analysis
			\item Findings include naming conventions embedded in IPv6 addresses that result in non-hierarchical patterns
		\end{itemize}
		
		{\bf Measuring IPv6 Transition Mechanisms}\hfill{\em July 2023 - November 2023}\\
		{\em Max Planck Institute for Informatics}\\
		{Advisor: Dr. Oliver Gasser}
		\begin{itemize}
			\item Measured and characterized transition technologies between IPv4 and IPv6 networks
			\item Findings include quantifying NAT64 deployment, the correctness of NAT64 configuration, and assessing the security of such infrastructure.
		\end{itemize}
		
		{\bf IPv6 Perspectives from Various Datasets} \hfill{\em August 2021 - October 2022}\\
		{\em Georgia Institute of Technology}\\
		{Advisors: Professor Paul Pearce, Professor Frank Li}
		\begin{itemize}
			\item Characterized IPv6 usage according to various metrics by comparing relevant datasets
			\item Datasets analyzed include: WHOIS records from Regional Internet Registries, routing data from Route Views and RIPE RIS, active IPv6 hitlists
			\item Developed new methodology for analyzing IPv6 apportionment
		\end{itemize}
		
		{\bf Distinguishing Organizations with IPv4 Scan Data} \hfill{\em July 2020 - May 2021}\\
		{\em University of Illinois at Urbana-Champaign}\\
		{Advisors: Professor Matthew Caesar}
		\begin{itemize}
			\item Analyzde Censys scanning data and WHOIS records for correlations between organizations and their IPv4 addresses
			% \item Specifically focus on trends across different types of organizations
		\end{itemize}
		
		% {\bf Aggregating Internet Data} \hfill{\em July 2020 - May 2021}\\
		% {\em University of Illinois at Urbana-Champaign}\\
		% {Advisor: Professor Matthew Caesar}
		% \begin{itemize}
			%     \item Synchronize various internet data sources to a searchable data structure with to deploy a toolkit usable to other researchers and engineers
			%     \item Sources include Route Views, CAIDA topologies, WHOIS data, and Shodan scanning data
			% \end{itemize}
		
		{\bf Circuit-Level Verification of P4 Programs} \hfill{\em January 2019 - May 2021}\\
		{\em University of Illinois at Urbana-Champaign}\\
		{Advisors: Professor William H. Sanders, Professor Matthew Caesar}
		\begin{itemize}
			\item Modeled data-plane programs as sequential circuits to be verified using hardware techniques including bounded model-checking
			\item Implemented with P4 language
		\end{itemize}
		
		{\bf Client Puzzles for State Exhaustion Attacks Resilience} \hfill{\em August 2018 - December 2018}
		{\em University of Illinois at Urbana-Champaign}\\
		{Advisor: Professor William H. Sanders}
		\begin{itemize}
			\item Evaluated client puzzles as a defense against Distributed Denial of Service (DDoS) attacks
			\item Implemented method of priority queuing requests determined by client puzzles in the TCP stack of the Linux Kernel
		\end{itemize}
		
	\end{rSection}
	% \vspace{20mm}
	%--------------------------------------------------------------------------------
	%    PROJECTS
	%-----------------------------------------------------------------------------------------------
	\begin{rSection}{Awards and Scholarships} 
		\textbf{•} {Community Engagement Award, \textit{School of Computer Science, Georgia Institute of Technology}} \hfill {\em 2023}
		\vspace{3mm}\\
		\textbf{•} {Graduate Research Fellowship Program (GRFP), \textit{National Science Foundation (NSF)}} \hfill {\em 2022}
		\vspace{3mm}\\
		\textbf{•} {Herbert P. Haley Fellowship, \textit{Georgia Institute of Technology}} \hfill {\em 2022}
		\vspace{3mm}\\
		\textbf{•} {President's Fellow, \textit{Georgia Institute of Technology}} \hfill {\em 2021}
		\vspace{3mm}\\
		\textbf{•} {Knights of St. Patrick Award, \textit{University of Illinois at Urbana-Champaign}} \hfill {\em 2021}
		\vspace{3mm}\\
		\textbf{•} {PricewaterhouseCoopers Grace Hopper Scholar} \hfill {\em 2018}
		\vspace{3mm}\\
		\textbf{•} {North Shore Community Service Award for Extra Effort} \hfill {\em 2017}
		
		\textbf{Travel Grants}
		\begin{itemize}
			\item ACM Internet Measurement Conference (IMC) 2023
			\item ACM SIGMETRICS Conference 2023
			\item ACM Internet Measurement Conference (IMC) 2022
		\end{itemize}
		
		
	\end{rSection}
	%--------------------------------------------------------------------------------
	%    ACTIVITIES
	%-----------------------------------------------------------------------------------------------
	\begin{rSection}{Professional Service}
		\textbf{•} Lead Student Organizer, ACM SIGCOMM 2021
		\begin{itemize}
			\item Presented Student Welcome Session for all students attending SIGCOMM 2021
			\item Collaborated with professionals in the SIGCOMM community to compile content relevant to students attending academic conferences
		\end{itemize}
		\textbf{•} Student Program Committee Volunteer, ACM SIGCOMM 2021
		\begin{itemize}
			\item Assisted in technical logistics of the 2-day-long review of paper submissions to SIGCOMM 2021
%			\item Ensured that no committee members with conflicts were present during paper reviews
		\end{itemize}
		
		\textbf{•} Reviewer, USENIX NSDI 2021
		
		\textbf{•} Student Organizer, ACM SIGCOMM 2020
		
		\textbf{•} Reviewer, ACM CCS 2020
		
	\end{rSection}


\begin{rSection}{Teaching}
	{\emph{Teaching Assistant} \bf CS8803 - Securing Internet Infrastructure} \hfill{\em January 2023 - May 2023}
	\vspace{-2mm}
	\begin{itemize}
		\item Instructor: Professor Cecilia Testart
	\end{itemize}
	% {\bf CUBE Consulting} \hfill{\em Jan. 2019 - May 2020}
	% \vspace{-2mm}
	% \begin{itemize}
		%     \item Consultant (2019-20) %%%%%
		
		% \end{itemize}
\end{rSection}
	\begin{rSection}{\bf Leadership and Extracurricular Experiences}
	\textbf{•} Founding Co-Organizer, Georgia Tech Networks Research Group \hfill{\em August 2022 - May 2023}
	\begin{itemize}
		\item Co-founded an interdisciplinary group of students and faculty in networks research
		\item Fostered social and professional connections between researchers at Georgia Tech
	\end{itemize}
	
	{\bf Society of Women Engineers, UIUC} \hfill{\em August 2017 - May 2021}
	\vspace{-2mm}
	\begin{itemize}
		\item President (2020-21)
		\item Treasurer (2019-20)%%%%%%
		% \item Webmaster (2018-19) %%%%
		% \item SWE 5k Chair (2017-18) %%%%%
	\end{itemize}
	% {\bf CUBE Consulting} \hfill{\em Jan. 2019 - May 2020}
	% \vspace{-2mm}
	% \begin{itemize}
		%     \item Consultant (2019-20) %%%%%
		
		% \end{itemize}
\end{rSection}
	%----------------------------------------------------------------------------------------
	%	INDUSTRY SECTION
	%----------------------------------------------------------------------------------------
	\begin{rSection}{Industry Experience}
		{\emph{Research Intern} \bf Censys} \hfill{\em May 2021 - August 2021}
		\vspace{-2mm}
		\begin{itemize}
			\item Designed and implemented methods of HTTP scanning to attribute domains, IP addresses, and certificates to organizations
		\end{itemize}
		{\emph{Software Engineering Intern} \bf Censys} \hfill{\em May 2020 - August 2020}
		\vspace{-2mm}
		\begin{itemize}
			\item Improved attribution system that utilizes Internet-wide scan data to associate assets including hosts, certificates, and domains, to customers
			\item Contributions include API development in Go and Python as well as database management
		\end{itemize}
		{\emph{Non-Volatile Memory Firmware Validation Intern} \bf Intel Co.} \hfill{\em May 2019 - August 2019}
		\vspace{-2mm}
		\begin{itemize}
			\item Developed Python scripts to collect data to standardize test system setup, including hardware and software specifications
			\item Reduced false-negatives on firmware validation tests
		\end{itemize}
		{\emph{Analyst Intern, Independent Contractor} \bf Bellwether Analytics} \hfill{\em June 2018 - March 2019}
		\begin{itemize}
%			\item Implemented small-scale data analysis for over 10,000 pharmaceutical records
			\item Built applications to create precise market landscapes which were used to advice R$\&$D departments of various pharmaceutical companies
%			\item Wrote JavaScript programs to collect and analyze data from specific public databases
%			\item Built GUI to make data analytics user-friendly
%			
		\end{itemize}
		
	\end{rSection}
	
	% \begin{rSection}{Conferences}
		% \textbf{•} WE20 SWE International Conference Attendee (2019), Anaheim, CA
		
		% \textbf{•} WE19 SWE International Conference Attendee (2019), Anaheim, CA
		
		% \textbf{•} WE18 SWE International Conference Attendee (2018), Minneapolis, MN
		
		
		% %--copy and paste this region  if you need more--
		% \end{rSection}
	%----------------------------------------------------------------------------------------
	%	SKILLS SECTION
	%----------------------------------------------------------------------------------------
	
	
	\begin{rSection}{Skills} 
		{C, C++, Python, Javascript, Assembly Language (x86)} 
	\end{rSection}
	
	
\end{document}----------------------------

